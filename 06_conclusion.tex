\selectlanguage{english}
\setlength{\parindent}{0.5cm}
\section{Conclusion}
This paper presented a novel method based on RNN for complaint classification.
After preprocessing including the rule-based normalization of the text, 
the neural network with the single or bi-directional RNN was trained to classify
the given sentence into nine compliant categories. 
By investigating several variations of the RNN models, 
we found that bi-directional LSTM with peephole was the best.
It achieved 0.904 F1-score.


The contribution of this paper is summarized as follows.
First, the denoising autoencoder with corrupted word (DACW) was
introduced to handle the unknown word in the training data.
The noisy inputs were automatically generated by duplicating
the sentence with replacement of the word with the unknown word.
To prevent the indicative keyword from being lost,
the replacement to the unknown word was limited to the words
only in the whitelist that is a list of less important words.
We confirmed that the model with DACW was significantly better
than the model without DACW. 
The denoising autoencoder with DACW can be applied to other NLP tasks. 
Second, sophisticated rules to clean up the Thai text were introduced
in the preprocessing step. Such rules were not applied in the previous study of the 
classification of the complaints in Thai \cite{assawinjaipetch-etal-2016-recurrent}.
The rules contributes to improve F1-score by 0.06.
Finally, the various RNN models were empirically evaluated.
We found that bi-directional RNN was better than single-direction
and the peephole gate was slightly better than the other configuration.


For the future work, we will further investigate
the LSTM architecture and the model structure for better performance.
Moreover, we are interested in applying DACW in various parts of the model
such as training of word embedding.
