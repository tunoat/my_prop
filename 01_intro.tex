\selectlanguage{english}
\setlength{\parindent}{0.5cm}
\section{Introduction}
Improvement of customer service is one of the most essential issues for many companies,
since it heavily influences customer's satisfaction.
The customer service department of big enterprise should handle a large amount
of requests from many customers.
Especially, it is important to deal with complaints from the customers.
The goal of this study is to automatically classify Thai texts of the customer's
complaints for efficient handling of them.


Since the idea of Deep Learning language model has been proposed
by Collobert et al. (2011) \cite{journals/corr/abs-1103-0398}, many studies have
reported that the word and phrase embeddings significantly boosted
the performance in many of natural language pro-cessing (NLP) tasks
(Socher et al., 2013; Zeng et el., 2014; Severyn and Moschitti, 2015; Lai et al., 2015).
\cite{socher2013recursive,zeng2014relation,severyn2015twitter,lai2015recurrent}
Especially, by applying famous word embedding techniques introduced by
Mikolov et al. (2013) \cite{mikolov2013distributed}, the language model became more precise.
However, a good word embedding may be difficult to obtain when the size of training data is
limited in some specific domains. The classification of customer complaint feedbacks
also suffers from a lack of the training data. In addition, customer's complaints
usually contain slang, typos, ungrammatical sentence and rich emotion.
Therefore, another difficulty of our task is preprocessing of Thai language.
Since there is no space boundary between the words in Thai, word segmentation is essential
for Thai text processing. However, the errors of the word segmentation especially for
ungrammatical sentences could lead to the significant decline of model performance.
In short, the model will have low tolerance against unknown word combination that
comes from ungrammatical input.


This paper proposes methods to improve the model for the complaint classification: 
denoising autoencoder with corrupted word and rule-based normalization in preprocessing.
Furthermore, several Recurrent Neural Network (RNN) architectures are empirically evaluated
in the task of complaint classification. The proposed methods significantly
improve F1-score by 0.06 from a baseline that applies only word
embedding with Bi-directional Long-Short Term Memory.
